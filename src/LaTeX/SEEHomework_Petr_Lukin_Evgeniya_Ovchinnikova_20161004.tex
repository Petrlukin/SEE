\documentclass[a4paper, 12pt]{article}
\usepackage{titling}
\usepackage{array}
\usepackage{booktabs}
\usepackage{enumitem}
\usepackage{graphicx}
\usepackage{hyperref}
\usepackage{amssymb}
\usepackage{listings}
\setlength{\heavyrulewidth}{1.5pt}
\setlength{\abovetopsep}{4pt}
\graphicspath{{.}}

\usepackage[margin=1in]{geometry}

% Must be after geometry
\usepackage{fancyhdr}
\pagestyle{fancy}
\fancyhf{}
\rhead{SEE Assignment 1}
\lhead{P.Lukin, E. Ovchinnikova}
\cfoot{\thepage}

\setlength{\droptitle}{-5em}

\title{Scientific Experimentation and Evaluation  \\
				Assignment: 1.1}
\author{Petr Lukin, Evgeniya Ovchinnikova}
\date{Lecture date: $04^{th}$ October 2016}

\begin{document}



\maketitle

\section{Construct the “simple robot” shown on the instruction sheet included in the LEGO construction
kit.}

We have assembled the robot in accordance to the instruction that is shown in Fig 1:
//TODO: take a photo of the assembled robot and attach here

\section{Equip your robot with a pen, such that you can mark the position of the robot at its stop
position.}
As it is shown in Fig.1, the robot is equipped with two light sensors. NXT light sensors emit bright red light. We have narrowed the light spot, so it would be small enough to mark the position of the robot with a sufficient precision. The advantage of this method is that the robot won't be moved in the moment of position measurement, as it can easily be moved when a pen touches the surface.

\section{Verify your robot is running LeJOS and program it to run for a certain
distance.}
We have installed and tested LeJOS. The test was performed with the following program for forward motion.

\begin{lstlisting}

package LegoNXT;

import lejos.nxt.LightSensor;
import lejos.nxt.Motor;
import lejos.nxt.SensorPort;
import lejos.robotics.navigation.DifferentialPilot;
import lejos.robotics.navigation.MoveController;
import lejos.util.Delay;

public class goStraight {

        public static void main(String[] args) {
                double arcRad = 30;
                double angle = 90;
                LightSensor lightSens1 = new LightSensor(SensorPort.S1);
                LightSensor lightSens2 = new LightSensor(SensorPort.S2);
                DifferentialPilot dp = new DifferentialPilot
                (MoveController.WHEEL_SIZE_NXT1, 18, Motor.B, Motor.C, 
                true);
                lightSens1.setHigh(100);
                lightSens2.setHigh(100);
                dp.setTravelSpeed(10);
                dp.travel(-Math.PI*2*arcRad*angle/360);        
                dp.stop();
                Delay.msDelay(10000);
        }
}

\end{lstlisting}
\section{What and how we are planning to do.}
We are planning to observe the motion model of Lego NXT robot with a differential drive. Based on these observations we will obtain a distribution of final poses and calculate motion parameters.\\
We are planning to estimate the position of the robot based on two light points emitted by light sensors. The starting position, the "garage", will also be marked as two points on the paper. Three series of experiments (around 30 times each) will be conducted: forward motion, left arc motion, right arc motion. For the forward motion the program above will be used for the left and right motions the following programs will be used:

\begin{lstlisting}
package LegoNXT;

import lejos.nxt.Motor;
import lejos.nxt.SensorPort;
import lejos.robotics.navigation.DifferentialPilot;
import lejos.robotics.navigation.MoveController;
import lejos.util.Delay;
import lejos.nxt.LightSensor;

public class goLeft {

        public static void main(String[] args) {
                double arcRad = 30;
                double angle = 90;
                LightSensor lightSens1 = new LightSensor(SensorPort.S1);
                LightSensor lightSens2 = new LightSensor(SensorPort.S2);
                DifferentialPilot dp = new DifferentialPilot
                (MoveController.WHEEL_SIZE_NXT1, 18, Motor.B, 
                Motor.C, true);

                lightSens1.setHigh(100);
                lightSens2.setHigh(100);
                dp.setTravelSpeed(10);
                dp.arc(-arcRad, angle);
                dp.stop();
                Delay.msDelay(10000);
        }
}

***

package LegoNXT;

import lejos.nxt.LightSensor;
import lejos.nxt.Motor;
import lejos.nxt.SensorPort;
import lejos.robotics.navigation.DifferentialPilot;
import lejos.robotics.navigation.MoveController;
import lejos.util.Delay;

public class goRight {

        public static void main(String[] args) {
                double arcRad = 30;
                double angle = 90;
                LightSensor lightSens1 = new LightSensor(SensorPort.S1);
                LightSensor lightSens2 = new LightSensor(SensorPort.S2);
                DifferentialPilot dp = new DifferentialPilot
                (MoveController.WHEEL_SIZE_NXT1, 18, Motor.B,
                 Motor.C, true);

                lightSens1.setHigh(100);
                lightSens2.setHigh(100);		
                dp.setTravelSpeed(10);
                dp.arc(arcRad, -angle);
                dp.stop();
                Delay.msDelay(10000);
        }
}

\end{lstlisting}
\section{Which difficulties we expect.}
First difficulty that we expect is an uneven work of the motors. NXT Lego robots usually have problems with a forward motions because of not very good synchronized motor work.

\end{document}
